%!TEX TS-program = xelatex
\documentclass[12pt,fleqn,titlepage,twoside,a4paper]{book}
% \usepackage{etex}
% \usepackage{amsfonts,amsmath,amssymb,graphicx}
% \usepackage{txfonts}
% \usepackage[centering,includeheadfoot,margin=1in]{geometry}
% \usepackage{tabvar}
\usepackage{arabxetex}
\newfontfamily{\arabicfont}[Script=Arabic,Scale=1.5]{Lateef}

\parindent = 0pt

\begin{document}

\section{arb}

 Genesis 11:1-9 http://live.bible.is/bible/ARBVDV/GEN/11

\begin{arab}[utf]

وَكَانَتِ ٱلْأَرْضُ كُلُّهَا لِسَانًا وَاحِدًا وَلُغَةً وَاحِدَةً. وَحَدَثَ فِي ٱرْتِحَالِهِمْ شَرْقًا أَنَّهُمْ وَجَدُوا بُقْعَةً فِي أَرْضِ شِنْعَارَ وَسَكَنُوا هُنَاكَ.وَقَالَ بَعْضُهُمْ لِبَعْضٍ: «هَلُمَّ نَصْنَعُ لِبْنًا وَنَشْوِيهِ شَيًّا». فَكَانَ لَهُمُ ٱللِّبْنُ مَكَانَ ٱلْحَجَرِ، وَكَانَ لَهُمُ ٱلْحُمَرُ مَكَانَ ٱلطِّينِ.وَقَالُوا: «هَلُمَّ نَبْنِ لِأَنْفُسِنَا مَدِينَةً وَبُرْجًا رَأْسُهُ بِٱلسَّمَاءِ. وَنَصْنَعُ لِأَنْفُسِنَا ٱسْمًا لِئَلَّا نَتَبَدَّدَ عَلَى وَجْهِ كُلِّ ٱلْأَرْضِ».فَنَزَلَ ٱلرَّبُّ لِيَنْظُرَ ٱلْمَدِينَةَ وَٱلْبُرْجَ ٱللَّذَيْنِ كَانَ بَنُو آدَمَ يَبْنُونَهُمَا.وَقَالَ ٱلرَّبُّ: «هُوَذَا شَعْبٌ وَاحِدٌ وَلِسَانٌ وَاحِدٌ لِجَمِيعِهِمْ، وَهَذَا ٱبْتِدَاؤُهُمْ بِٱلْعَمَلِ. وَٱلْآنَ لَا يَمْتَنِعُ عَلَيْهِمْ كُلُّ مَا يَنْوُونَ أَنْ يَعْمَلُوهُ.هَلُمَّ نَنْزِلْ وَنُبَلْبِلْ هُنَاكَ لِسَانَهُمْ حَتَّى لَا يَسْمَعَ بَعْضُهُمْ لِسَانَ بَعْضٍ».فَبَدَّدَهُمُ ٱلرَّبُّ مِنْ هُنَاكَ عَلَى وَجْهِ كُلِّ ٱلْأَرْضِ، فَكَفُّوا عَنْ بُنْيَانِ ٱلْمَدِينَةِ،لِذَلِكَ دُعِيَ ٱسْمُهَا «بَابِلَ» لِأَنَّ ٱلرَّبَّ هُنَاكَ بَلْبَلَ لِسَانَ كُلِّ ٱلْأَرْضِ. وَمِنْ هُنَاكَ بَدَّدَهُمُ ٱلرَّبُّ عَلَى وَجْهِ كُلِّ ٱلْأَرْضِ.
\end{arab}

\section{arz}

 Genesis 11:1-9 https://www.bible.com/bible/2429/GEN.11.EAT

\begin{arab}[utf]

\section*{\textarab[utf]{قصة برج بابل}}


في الأول كان البشر بيتكلمو بلسان واحد وبيستخدمو كلمات معروفة بين كل الناس. 2ولما رحل الناس من الشرق، استقروا في أرض بابل. 3‏-4وبعدين قالوا لبعض: «تعالوا نبني مدينة وبرج توصل قمته للسماء. ونستخدم في بناء البرج القطران وطوب القش المُجفف بالنار بدل من الحجارة والطين. فتنتشر شهرتنا بين الناس، ومنتفرقش على الأرض ولا يتنسىَ اسمنا.» 5لكن لما شاف ربنا الناس بيبنو المدينة والبرج، 6قالَ: «الناس دي بيعملو كده لأنهم شعب واحد وبيتكلموا بنفس اللسان. وبعد كده ممكن يعملوا أي حاجة هم عايزينها. 7فهَنزِّل عليهم اللي يبلبل لسانهم، فميقدروش يفهموا بعضهم بسبب اختلاف لغاتهم.» 8‏-9فَبلبل الله لسانهم، وشتتهم من هناك في الأرض كلها. وده سبب تسمية مدينة بابل بالأسم ده.

\end{arab}


\section{ayh}

Genesis 11:1-9 https://www.bible.com/bible/2428/GEN.11.HAT

\begin{arab}[utf]
\section*{\textarab[utf]{ ألله يُبلبل لغة الناس ويُشتّتهم}}

وكان الناس كلهم يتكلمون لغة واحدة ويستعملون كلمات يعرفونها الجميع. 2فرحلو إلى الشرق. وعندما وصلو وادي شنعار في أرض بابل، إستقرّو هناك. 3فقال بعضهم لبعض، "قومو بانعمل مدر وبانحرقه بالنار." فعملو المدر بدل الحصاء، والدامر بدل الطين. 4بعدين قالو، "يا جماعة، بانبني لأنفسنا مدينة عظيمة وعمارة يصل طولها إلى السماء. بانعمل لأنفسنا إسم علشان ما نتشتّت في الأرض." 5فشاف ألله عزوجل المدينة والعمارة إلي يبنوهن الناس. 6فقال الله: "لأنهم قوم واحد ويتكلمون لغة وحدة، ما بايكون صعب عليهم عمل أي شيء ثاني ينوون عمله. 7فننزل وُنبلبل لغتهم علشان مايفهم بعضهم كلام بعض." 8فبلبل ألله سبحانه وتعالى لغتهم، وشتّتهم من أرض بابل الى كل الأرض. فتوقّفو من بناء المدينة والعمارة. 9لذا سُمّيت ذي الأرض أرض بابل لأن الله القدير بلبل لغة الناس كلهم هناك. ومن هُناك شتّتهم في جميع أنحاء الأرض.

\end{arab}

\section{azb}

 Genesis 11:1-9 http://live.bible.is/bible/AZBEMV/GEN/11

\begin{arab}[utf]

\section*{\textarab[utf]{ بابئل بورجو }}

بوتون دونیادا بئر دئل و بئر دانیشیق وار ائدی.ائنسانلار شرقدن کؤچدوکلری زامان شئنعار تورپاغیندا بئر اووا تاپدیلار و اورادا ساکئن اولدولار.سونرا بئربئرلرئنه ددئلر: «گلئن کَرپئج دوزَلدک و اونو یاخشی پئشئرک.» اونلار داش عوضئنه کَرپئج و پالچیق عوضئنه قَطران گؤتوردولر.گئنه ددئلر: «گلئن اؤزوموزه بئر شهر و باشی گؤیلره چاتان بئر بورج تئکئب آد چیخارداق، یوخسا بوتون یِر اوزونه داغیلاجاغیق.»

رب آشاغی اِندی کی، بشر اؤولادلارینین تئکدئیی شهر و بورجو گؤرسون.رب ددی: «باخ، اونلارین هامیسی بئر خالقدیر و بئر دئللری وار. و بو اونلارین اتدئکلرئن باشلانیشی‌دیر. و ائندی اتمک ائستدئکلری هچ ائش اونلارا قیری-مومکون اولمایاجاق.گلئن آشاغی اِنئب، اورادا اونلارین دئلئنی قاریشدیراق کی، بئربئرلرئنئن دئلئنی باشا دوشمه‌سئنلر.»رب اونلاری اورادان بوتون یِر اوزونه یایدی؛ و اونلار شهری تئکمکدن اَل چکدئلر.بونا گؤره ده او یرئن آدینا بابئل ددئلر، چونکی اورادا رب بوتون دونیاداکی آداملارین دئلئنی قاریشدیردی و اورادان اونلاری بوتون یِر اوزونه یایدی.

\end{arab}

\section{ckb}

 Genesis 11:1-9 https://live.bible.is/bible/CKBIBS/GEN/11

\begin{arab}[utf]

\section*{\textarab[utf]{ قوللەی بابل }}

1هەموو جیهان یەک زمان و یەک جۆر قسەکردنیان هەبوو.ئینجا کاتێک بۆ ڕۆژهەڵات کۆچیان کرد، دەشتاییەکیان لە خاکی شینعار
دۆزییەوە و لەوێ نیشتەجێ بوون.

3ئینجا بە یەکتریان گوت: «وەرن، با خشت دروستبکەین و تەواو سووری بکەینەوە.» ئیتر لە جیاتی بەرد خشتیان بەکارهێنا، قیڕیش لە جیاتی قوڕ.
هەروەها گوتیان: «وەرن، با شارێک بۆ خۆمان دروستبکەین لەگەڵ قوللەیەک کە سەری لە ئاسمان بدات. ناوێکیش بۆ خۆمان دروستدەکەین، تاکو بەسەر هەموو ڕووی زەویدا پەرت نەبین.»

5بەڵام یەزدان بۆ بینینی ئەو شار و قوللەیەی کە ئادەمیزاد دروستی دەکرد هاتە خوارەوە.یەزدان فەرمووی: «ئەگەر هەموویان یەک گەل و یەک زمان بن، ئەوە سەرەتای دەستپێکردنیانە، هیچ شتێک لەلایان مەحاڵ نابێت، لەوەی مەبەستیانە بیکەن.

وەرن با بچینە خوارەوە و لەوێ زمانیان بشێوێنین، بۆ ئەوەی کەسیان لە زمانی ئەوی دیکەیان نەگات.»

8ئینجا یەزدان لەوێوە بەسەر هەموو زەویدا پەرتی کردن. ئیتر وازیان لە بنیادنانی شارەکە هێنا.لەبەر ئەمە ناونرا بابل
، چونکە لەوێدا یەزدان زمانی هەموو زەوی شێواند. هەر لەوێشەوە یەزدان بەسەر هەموو ڕووی زەویدا پەرتی کردن.

\end{arab}


\section{fub}

Genesis 11:1-9 https://www.bible.com/en-GB/bible/1707/GEN.11....

\begin{arab}[utf]
\section*{\textarab[utf]{ سُودُ تٛوْندُ حَا بَابِلَ }}

ندٜرْ وَکَّتِ مَاجُمْ يِمْࢡٜ ڢُو طٛنّٛ مبٛلْوَ طٜمْنغَلْ غٛوتَلْ عٜ بٛلّٜ غٛوتٜ؞ 2ندٜ يِمْࢡٜ ندِلِّ غَلْ ڢُونَانغٜ، ࢡٜتَوِ وَادِوٛلْ ندٜرْ لٜسْدِ سِنٜعَرْ، ࢡٜنجٛوطِ تٛنْ؞ 3ࢡٜمبِعْمبِعْتِرِ‏: «مبَطٜنْ بِرِکْجٜ، نغُلٜنْ طٜ حَا طٜ شَاتَ؞» ࢡٜࢩِࢡْرِ بٜى مَاجٜ بَنَ بٜى کَاعٜ، ࢡٜکٛوعِ تَارِ لٜسْدِ بٛو نغَمْ تَکِّنْدِرْغٛ طٜ؞ 4ࢡٜمبِعِ‏: «ندِلّٜنْ‏! ࢩِࢡٜنْ بٜرْنِوٛلْ عٜ سُودُ تٛوْندُ، حَا حٛورٜ سُودُ ندُعُ يٛتّٛ عَسَمَ، نغَمْ حَا مبَطَنٜنْ کٛعٜ مٜىٰطٜنْ عِنْدٜ، تَا عٜنْ شَنغْکِتٛ ندٜرْ دُنِيَارُ ڢُو؞» 5جَوْمِرَاوٛ جِݠِّ نغَمْ لَارُغٛ بٜرْنِوٛلْ عٜ سُودُ تٛوْندُ، کٛ ࢡِࢡّٜ عَادَمَ ࢩِࢡَتَ؞ 6عٛوِعِ‏: «ندَا، ࢡٜ عُمَّاتٛورٜ وٛورٜ، ࢡٜطٛنْ مبٛلْوَ طٜمْنغَلْ غٛوتَلْ؞ کٛ ࢡٜنغَطِ طٛ، طُمْ ڢُطّٛودٜ تَوٛنْ؞ جٛنْتَ کَمْ، وَلَا کٛ حَطَتَ ࢡٜ وَطُغٛ کٛ ࢡٜنُڢٛتٛ ڢُو؞ 7ندِلّٜنْ، عٜنْ نجِݠّٛ، عٜنْ نجِيࢡَ وٛلْدٜ مَࢡّٜ، تَا ࢡٜݠَامْݠَامْتِرَ‏!» 8جَوْمِرَاوٛ سَنغْکِتِ ࢡٜ دِغَ تٛنْ ندٜرْ دُنِيَارُ ڢُو، ࢡٜعَشِّ ࢩِࢡُغٛ بٜرْنِوٛلْ؞ 9نغَمْ مَاجُمْ نغٛلْ عِنْدِنَامَ بَابِلَ، نغَمْ حَا تٛنْ جَوْمِرَاوٛ جِيࢡِ وٛلْدٜ يِمْࢡٜ دُنِيَارُ ڢُو عٜ دِغَ تٛنْ بٛو عٛسَنغْکِتِ ࢡٜ ندٜرْ دُنِيَارُ ڢُو؞

\end{arab}


\section{hau}

 Genesis 11:1-9

\begin{arab}[utf]
\section*{\textarab[utf]{ حَصُومِيَا تَ بَابِيلَ }}

1أَ دَا ثَنْ، مُتَنٜىٰنْ دُونِيَا سُنَدَ يَرٜىٰ طَيَ نٜىٰ تَكْ، كَلْمُواْمِنْسُ كُمَ إِرِ طَيَ نٜىٰ؞ 2أَنَانً سَعَدَّ مُتَنٜىٰ سُكٜىٰ تَيَطُوَ دَغَ يَنْكِنْ غَبَسْ، سَيْ سُكَ سَامِ ڢِيلِ عَڧَسَرْ شِنَرْ، سُكَ ذَوْنَ عَثَنْ؞ 3سَيْ سُكَثٜىٰ وَجُونَ ‏«‏كُذُواْ، مُيِ بُلُواكْ، مُغَسَسُ سُواْسَيْ؞»‏ سُكَيِ عَيْكِ دَ بُلُواكْ نَلَاكَ مَيْمَكُوانْ دُوڟٜىٰ، مَنْݣُولْتَ كُمَ مَيْمَكُوانْ لَاكَ؞ 4سَيْ سُكَثٜىٰ ‏«‏ذُواْ مُغِنَ وَكَنْمُ بِرْنِ دَ حَصُومِيَا وَدَّ رُڢِنْتَ ذَاتَكَيْ ثَنْ ثِكِنْ سَمَ، دُواْمِنْ مُيِ وَكَنْمُ سُونَ، دُواْمِنْ كَدَ مُوَرْوَڟُ ذُوَا كُواْعِنَ أَ ڢُسْكَرْ دُونِيَا؞»‏ 5سَيْ يَهْوٜىٰهْ يَسَوْكُواْ دُواْمِنْ يَدُوبَ بِرْنِ دَ حَصُومِيَرْدَ یَنْ أَدَمْ سُكٜىٰ غِنَاوَا؞ 6سَعَنً يَهْوٜىٰهْ يَثٜىٰ ‏«‏غَاشِ، سُو جَمَعَ طَيَ نٜىٰ، دُكَنْسُ كُمَ يَرٜىٰ طَيَ نٜىٰ؞ وَنَّنْ كُوَ أَبُ نَڢَرْكُواْ نٜىٰ كَطَيْ نَ أَبُبُوَنْ دَ ذَاسُ عِيَيِ، بَابُ كُمَ أَبِنْدَ سُكَيِ نِيَّرْيِ دَ ذَيْ غَغَرٜىٰسُ؞ 7ذُواْ مُسَاكٜىٰ سَوْكَ، مُدَاغُلَ يَرٜىٰنْسُ أَوُرِنْ، دُواْمِنْ كَدَ سُغَانٜىٰ مَغَنَرْ جُونَ؞»‏ 8حَكَ كُوَ يَهْوٜىٰهْ يَا وَرْوَڟَاسُ كُواْعِنَ أَ ڢُسْكَرْ دُونِيَا، سُكَ كُمَ دَيْنَ غِنَا بِرْنِنْ؞ 9سَبُواْدَ حَكَ عَكَ كِرَا بِرْنِنْ بَابِيلَ،*1 دُواْمِنْ أَوُرِنْ نٜىٰ يَهْوٜىٰهْ يَدَاغُلَ يَرٜىٰنْ دُكَنْ دُونِيَا؞ دَغَ وُرِنْ نٜىٰ يَهْوٜىٰهْ يَوَرْوَڟَاسُ كُواْعِنَ أَ ڢُسْكَرْ دُونِيَا؞ 

\end{arab}

\section{pbu}

Genesis 11:1-9 https://www.bible.com/en-GB/bible/2510/GEN.11.PYPB

\begin{arab}[utf]
% \chapter*{\textarab[utf]{ حِكَم من تَجمـيعي }}
\section*{\textarab[utf]{  د بابل برج}}

1په شروع کښې، د ټولې دُنيا خلقو صِرف يوه ژبه وئيله او د يو بل په خبرو پوهېدل. 2چې څنګه دوئ د نمرخاتۀ غاړې نه روان شول، چې چرته دوئ اوسېدل، نو د بابل په مُلک کښې يو هوار مېدان ته راغلل او هلته دېره شول. 3دوئ يو بل ته ووئيل، ”راځئ چې خَښتې جوړې کړُو او په اور يې پخې کړُو.“ نو داسې د دوئ سره د آبادۍ دپاره خَښتې او د دې د انښلولو دپاره تارکول وُو. 4نو دوئ ووئيل، ”راځئ چې اوس يو ښار او دومره لوئ برج جوړ کړُو چې آسمان ته ورسيږى، نو مونږ به مشهور شُو او په ټوله دُنيا کښې به خوارۀ وارۀ هم نۀ شُو.“ 5بيا مالِک خُدائ د ښار او هغه برج کتلو له راکوز شو چې بنيادمو جوړ کړے وو، 6نو هغۀ وفرمائيل، ”وګورئ، دا ټول يو خلق دى او دوئ هم يوه ژبه وائى، او د دې نه پس چې بيا دوئ څۀ کول هم وغواړى نو کولے به يې شى، 7راځئ چې لاندې ورشُو او د دوئ ژبې ګډې وډې کړُو نو د يو بل په خبرو به نۀ پوهيږى.“ 8نو مالِک خُدائ دوئ په ټوله زمکه خوارۀ وارۀ کړل او دوئ ښار جوړول پرېښودل. 9نو د هغه ښار نوم بابل شو، ځکه چې مالِک خُدائ هلته د خلقو ژبه ګډه وډه کړه او د هغه ځائ نه يې په ټوله زمکه خوارۀ وارۀ کړل.

\end{arab}

\section{pes}

Genesis 11:1-9 http://live.bible.is/bible/PESNMV/GEN/11

\begin{arab}[utf]
% \chapter*{\textarab[utf]{ حِكَم من تَجمـيعي }}
\section*{\textarab[utf]{  برج بابِل}}

۝١  و اما، تمامی زمین را یک زبان و یک گفتار بود. ۝٢ و چون مردم از مشرق کوچ می‌کردند، در سرزمین شِنعار دشتی هموار یافتند و در آنجا ساکن شدند. ۝٣ آنان به یکدیگر گفتند: «بیایید خشتها بزنیم و آنها را خوب بپزیم.» ایشان را خشت به جای سنگ و قیر به جای ملات بود. ۝٤ آنگاه گفتند: «بیایید شهری برای خود بسازیم و برجی که سر بر آسمان ساید، و نامی برای خود پیدا کنیم، مبادا بر روی تمامی زمین پراکنده شویم.» ۝٥ اما خداوند فرود آمد تا شهر و برجی را که بنی‌آدم بنا می‌کردند، ببیند. ۝٦ و خداوند گفت: «اینک آنان قومی یگانه‌اند و ایشان را جملگی یک زبان است و این تازه آغازِ کارِ آنهاست؛ و دیگر هیچ کاری که قصد آن بکنند، از ایشان بازداشته نخواهد شد. ۝٧ اکنون فرود آییم و زبان ایشان را مغشوش سازیم تا سخن یکدیگر را درنیابند.» ۝٨ پس خداوند آنان را از آنجا بر روی تمامی زمین پراکنده ساخت و از ساختن شهر بازایستادند. ۝٩ از این رو آنجا را بابِل نامیدند، زیرا در آنجا خداوند زبان همۀ جهانیان را مغشوش ساخت. از آنجا، خداوند ایشان را بر روی تمامی زمین پراکنده کرد.

\end{arab}


\section{prs}

Genesis 11:1-9 http://live.bible.is/bible/PRSGNN/GEN/11

\begin{arab}[utf]
\section*{\textarab[utf]{ برج بابل}}

در آن زمان مردم سراسر جهان فقط یک زبان داشتند و کلمات آن ها یکی بود. 2وقتی که از مشرق کوچ می کردند، به زمین همواری در سرزمین شِنعار رسیدند و در آنجا ساکن شدند. 3آن ها به یکدیگر گفتند: «بیائید خشت بسازیم و آن ها را خوب پخته کنیم.» آن ها بجای سنگ از خشت و بجای گچ از قیر استفاده کردند. 4پس به یکدیگر گفتند: «بیائید شهری برای خود بسازیم و برجی بنا کنیم که سرش به آسمان برسد و بدینوسیله نام خود را جاودان بسازیم. مبادا در روی زمین پراگنده شویم.»

5بعد از آن خداوند پائین آمد تا شهر و برجی را که آن مردم ساخته بودند، ببیند. 6آنگاه فرمود: «حالا دیگر تمام این مردم متحد شدند و زبان شان هم یکی است. این هنوز شروع کار آن ها است. و هیچ کاری نیست که انجام آن برای آن ها غیر ممکن باشد. 7پس پائین برویم و وحدت زبان آن ها را از بین ببریم تا زبان یکدیگر را نفهمند.» 8پس خداوند آن ها را در سراسر روی زمین پراگنده کرد و آن ها نتوانستند آن شهر را بسازند. 9اسم آن شهر را بابل گذاشتند، چونکه خداوند در آنجا وحدت زبان تمام مردم را از بین برد و آن ها را در سراسر روی زمین پراگنده کرد.

\end{arab}

\section{shu}

Genesis 11:1-9 https://www.bible.com/bible/502/GEN.11...

\begin{arab}[utf]
\section*{\textarab[utf]{ بُنى قصِر بَابِل}}

زَمَانْ، كُلَّ النَّاسْ قَاعِدِينْ يِحَجُّوا بِلُغَّةْ وَاحِدَةْ وَ بِلِسَانْ وَاحِدْ. 2وَ النَّاسْ حَوَّلَوْا مِنْ صَبَاحْ وَ لِقَوْا سَهَلَةْ وَسِيعَةْ فِي بَلَدْ شِنْعَارْ وَ سَكَنَوْا فَوْقهَا. 3وَ قَالَوْا أَمْبَيْنَاتْهُمْ: «تَعَالُوا! نَخْدُمُوا وَ نِسَلُّلُوا دِرِنْقَيْل وَ نُتُشُّوهُمْ.»
وَ خَلَاصْ، فِي بَدَلْ الْحَجَرْ أَسْتَعْمَلَوْا الدِّرِنْقَيْل وَ أَمْبَيْنَاتْ الدِّرِنْقَيْل صَبَّوْا الزِّفْت بَدَلْ الطِّينَةْ. 4خَلَاصْ قَالَوْا: «تَعَالُوا نِسَوُّوا لَيْنَا مَدِينَةْ وَ نَبْنُوا قَصِرْ طَوِيلْ الرَّاسَهْ يَلْحَقْ السَّمَاءْ أَشَانْ أُسُمْنَا يَبْقَى طَالِعْ وَ مَا نِشِتُّوا فِي الْأَرْض.»
5وَ وَكِتْ النَّاسْ يِدَوْرُوا يَلْحَقَوْا السَّمَاءْ، مِنْ فَوْق اللّٰهْ شَافَاهُمْ وَ شَافْ الْمَدِينَةْ وَ الْقَصِرْ الْبَنِي آدَم قَاعِدِينْ يَبْنُوهْ. 6وَ اللّٰهْ قَالْ: «دَاهُو هُمَّنْ شَعَبْ وَاحِدْ وَ لُغِّتْهُمْ كُلَ وَاحِدَةْ. وَ دِي بِدَايَةْ خِدْمِتْهُمْ. كَنْ دَوَّرَوْا يِسَوُّوا خِدْمَةْ آخَرَةْ كُلَ، مَا فِي شَيّءْ يَدْحَرْهُمْ! 7خَلِّي نِبَرْجُلُوا لُغِّتْهُمْ أَشَانْ مَا يِلْفَاهَمَوْا أَمْبَيْنَاتْهُمْ.»
8بِمِثِلْ دَا، اللّٰهْ شَتَّتَاهُمْ فِي كُلَّ الْأَرْض وَ هُمَّنْ خَلَّوْا مِنْهُمْ بُنَى هَنَا الْمَدِينَةْ. 9وَ بَيْدَا بَسْ، الْمَدِينَةْ سَمَّوْهَا بَابِلْ (مَعَنَاتَهْ بَرْجَلَةْ) أَشَانْ اللّٰهْ بَرْجَلْ لُغَّةْ كُلَّ النَّاسْ وَ شَتَّتَاهُمْ فِي كُلَّ الْأَرْض.
\end{arab}

\section{sus}

Genesis 11:1-9 https://www.bible.com/bible/1379/GEN.11.SOSO

\begin{arab}[utf]
\section*{\textarab[utf]{ شُي مَسُنبُشِ كِ نَشّ}}

1نَ تّمُي دُنِحَ مِشِ بِرِن نُ شُي كٍرٍن نَن قَلَمَ. 2مِشِ ندٍيٍ نَشَ كٍلِ، عٍ سِفَ سٌفٍتٍدٍ مَبِرِ، عٍ فٍيَ تٌ سِنَرِ بْشِ مَ. عٍ نَشَ سَبَتِ كّنّ مَ نَشَن نَ فُلُنبَ كُي. 3عٍ نَشَ عَ قَلَ عٍ بٌورٍ بّ، «وٌ قَ، وٌن شَ بِرِكِ بْنبْ، وٌن شَ عٍ فَن.» عٍ نَشَ بِرِكِ قِندِ فّمّ حْشْي رَ. عٍ نَشَ مْتَ قِندِ دٌلٍ حْشْي رَ. 4نَ شَنبِ عٍ نَشَ عَ قَلَ، «وٌن وَلِ سُشُ، وٌن شَ تَا تِ وٌن يّتّ بّ، عَ نُن بَنشِ بٍلٍبٍلٍ نَشَن تٍمَ هَن كٌورٍ مَ، عَلَكٌ وٌن شِلِ شَ فبٌ. نَ نَ عَ رَ، وٌن مُ لْيمَ وٌن بٌورٍ مَ.»

5كْنْ عَلَتَلَ نَشَ فٌرٌ نَ تَا مَتٌدٍ، عَ نُن نَ بَنشِ بٍلٍبٍلٍ. 6عَلَتَلَ نَشَ عَ مَسٍن، «شَ يِ مِشِيٍ بَرَ كَقُ عٍ بٌورٍ مَ يِ وَلِ رَبَدٍ، عٍ نَ شُي كٍرٍن قَلَ، وَلِ بِرِن عٍ وَ مَ نَشَن نَبَقٍ، عَ سْونّيَمَ نّ. 7وٌن شّي، وٌن شَ فٌرٌ عٍ شَ شُي مَسُنبُدٍ عَلَكٌ عٍ نَشَ عٍ بٌورٍ وْيّن شُي قَهَامُ.» 8نَ كُي عَلَتَلَ نَشَ نَ فَلِ رَيٍنسٍن يّ. نَ بَنشِ تِقٍ نَشَ دَن. 9نَ يِرٍ شِلِ نَشَ سَ بَبٍلِ، بَرِ مَ عَلَتَلَ دُنِحَ مِشِيٍ شُييٍ رَوُيَ مّننِ نّ، عَ قَ عٍ رَيٍنسٍن يّ دُنِحَ بِرِن مَ.


\end{arab}


\section{uig}

Genesis 11:1-9 http://live.bible.is/bible/UIGUMK/GEN/11

\begin{arab}[utf]

دەسلەپتە، يەر يۈزىدە بىرلا تىل بار ئىدى، ئىنسانلار پەقەت بۇ تىلدىلا سۆزلىشەتتى.ئۇلار شەرق تەرەپكە كۆچۈۋاتقىنىدا، بابىلوندىكى بىر تۈزلەڭلىككە كېلىپ ئورۇنلاشتى.ئۇلار ئۆزئارا: - خىش قۇيۇپ، ئوتتا پىشۇرايلى! - دەپ مەسلىھەتلەشتى. بۇنىڭ بىلەن، ئۇلار قۇرۇلۇشتا تاشنىڭ ئورنىغا خىش، لاينىڭ ئورنىغا قارا ماي ئىشلىتىدىغان بولدى.ئاندىن، ئۇلار: - بىر شەھەر بىنا قىلايلى! شەھەردە ئاسمانغا تاقاشقۇدەك بىر مۇنار ياساپ، ئۆز نامىمىزنى چىقىرىپ، تەرەپ-تەرەپكە تارقىلىپ كېتىشىمىزدىن ساقلىنايلى! - دېيىشتى.پەرۋەردىگار ئىنسانلارنىڭ بىنا قىلىۋاتقان شەھەر ۋە مۇنارىنى كۆرگىلى چۈشتى.پەرۋەردىگار: - ئۇلار بىرلىشىپ بىر قوۋم بولۇپ، بىر تىلدا سۆزلىشىدىكەن. بۇ پەقەتلا ئۇلار قىلماقچى بولغان ئىشنىڭ باشلىنىشى! بۇنىڭدىن كېيىن، ئۇلار نېمە قىلىشنى خالىسا، شۇنى قىلالايدۇ.بىز چۈشۈپ، ئۇلارنىڭ تىلىنى قالايمىقانلاشتۇرۇۋېتىپ، ئۇلارنى بىر-بىرىنىڭ مەقسىتىنى ئۇقالمايدىغان قىلىپ قويايلى، - دېدى.بۇنىڭ بىلەن، پەرۋەردىگار ئۇلارنى پۈتۈن دۇنياغا تارقىتىۋەتتى. ئۇلار شەھەر بىنا قىلىش قۇرۇلۇشىنى توختاتتى.شۇڭا، بۇ شەھەرنىڭ نامى بابىل دەپ ئاتالدى. چۈنكى، پەرۋەردىگار ئۇ يەردە ئىنسانلارنىڭ تىللىرىنى قالايمىقانلاشتۇرۇۋېتىپ، ئۇلارنى دۇنيانىڭ ھەرقايسى جايلىرىغا تارقىتىۋەتكەنىدى.

\end{arab}


\section{urd}

Genesis 11:1-9 http://live.bible.is/bible/URDURV/GEN/11

\begin{arab}[utf]

% \addfontfeature{RawFeature={+cv82=2}}
% \setmainfont[Script=Arabic,Scale=1,CharacterVariant={82:2}]{Lateef Book}
% \section*{\textarab[utf]{  برج بابِل }}
%\addfontfeature{CharacterVariant={82:1}}
%\addfontfeature{CharacterVariant={48:1}} % urd
\fontspec{Lateef}[Script=Arabic,Language=Urdu,Scale=1.5]


۝۱ اور تمام زمِین پر ایک ہی زُبان اور ایک ہی بولی تھی۔ ۝۲  اور اَیسا ہُؤا کہ مشرِق کی طرف سفر کرتے کرتے اُن کو مُلکِ سِنعار میں ایک میدان مِلا اور وہ وہاں بس گئے۔ ۝۳ اور اُنہوں نے آپس میں کہا آؤ ہم اِینٹیں بنائیں اور اُن کو آگ میں خُوب پکائیں ۔ سو اُنہوں نے پتّھر کی جگہ اِینٹ سے اور چونے کی جگہ گارے سے کام لِیا۔ ۝۴ پِھر وہ کہنے لگے کہ آؤ ہم اپنے واسطے ایک شہر اور ایک بُرج جِس کی چوٹی آسمان تک پُہنچے بنائیں اور یہاں اپنا نام کریں ۔ اَیسا نہ ہو کہ ہم تمام رُویِ زمِین پر پراگندہ ہو جائیں۔ ۝۵ اور خُداوند اِس شہر اور بُرج کو جِسے بنی آدم بنانے لگے دیکھنے کو اُترا۔ ۝۶ اور خُداوند نے کہا دیکھو یہ لوگ سب ایک ہیں اور اِن سبھوں کی ایک ہی زُبان ہے ۔ وہ جو یہ کرنے لگے ہیں تو اب کُچھ بھی جِس کا وہ اِرادہ کریں اُن سے باقی نہ چُھوٹے گا۔ ۝۷ سو آؤ ہم وہاں جا کر اُن کی زُبان میں اِختلاف ڈالیں تاکہ وہ ایک دُوسرے کی بات سمجھ نہ سکیں۔ ۝۸ پس خُداوند نے اُن کو وہاں سے تمام رُویِ زمِین میں پراگندہ کِیا سو وہ اُس شہر کے بنانے سے باز آئے۔ ۝۹ اِس لِئے اُس کا نام بابل ہُؤا کیونکہ خُداوند نے وہاں ساری زمِین کی زُبان میں اِختلاف ڈالا اور وہاں سے خُداوند نے اُن کو تمام رُویِ زمِین پر پراگندہ کِیا۔


Character Variant

\addfontfeature{CharacterVariant={82:0}} 0 (Sindhi cv82=1)
۴۵۶۷

\addfontfeature{CharacterVariant={82:1}} 1 (Urdu cv82=2)
۴۵۶۷

\addfontfeature{CharacterVariant={82:2}} 2 (default cv82=0)
۴۵۶۷

\addfontfeature{CharacterVariant={82:3}} 3 (Rohingya cv82=4)
۴۵۶۷

\addfontfeature{CharacterVariant={82:4}} 4 (default cv82=0)
۴۵۶۷

Raw Feature

\addfontfeature{RawFeature={+cv82=0}} 0 (Sindhi cv82=1)
۴۵۶۷

\addfontfeature{RawFeature={+cv82=1}} 1 (Urdu cv82=2)
۴۵۶۷

\addfontfeature{RawFeature={+cv82=2}} 2 (default cv82=0)
۴۵۶۷

\addfontfeature{RawFeature={+cv82=3}} 3 (Rohingya cv82=4)
۴۵۶۷

\addfontfeature{RawFeature={+cv82=4}} 4 (default cv82=0)
۴۵۶۷

\end{arab}


\section{apc}

Psalm 23 https://www.bible.com/bible/2810/PSA.23.LRT

\begin{arab}[utf]
\section*{\textarab[utf]{دعاء الثقة بالله للنبي داود (ع) }}

الله وكيلي هو مْراعيني،
مَعُه ما بْعوز شي لأنّه مْكَفّيني
2عأَراضي خَضرا بْياخِدني، بِرَيّحني،
وجنب المَيّ الهادية بْيِهْديني.
3بِرَيِّح نَفسي بْيِنْعِشْني،
ولَسَبيل البِرّ بْيِرْشِدْني،
إكراماً لإسمُه الصّادق الأمين.
4وحتّى لو مْشيت بالوادي المُظْلم المُميت،
ما بْخاف من الأذى أبداً، لأنّك مَعي يا معين.
بْتِحْميني مِتل الرّاعي بْعَصاه، وبْتِهْديني
5سُفْرِة مْلوك بِتْحضِّرلي، علناً إدّام عَدوّيني،
بِتْكَرِمّني آخِر كَرَم وبْأعلى تِشْريف بْتِدْعيني
6نَعَم، الخير والوَفا بِرافقوني حتّى مَماتي،
ورَح إسكُن في بيت الله كلّ إيام حياتي.

\end{arab}


\section{mzn}

Ruth 1:1-5  https://www.bible.com/bible/2755/RUT.1.MAZ

\begin{arab}[utf]
\section*{\textarab[utf]{کوچ کَشی }}

1‏-2خَله ساله پیش که اسرائیلِ مملِکِت هیچ پادشاهی نِداشته، اسرائیلِ دِله قحطی و خِشکسالی بیَمو. اَتّا مردی که وِنه نوم اِلیمِلِک و وِنه تِبار اِفراته‌ای بی‌یِه، با شه زِنا نَعومی و دِ تا ریکا (مَحلون و کِلیون)، بِیت‌لِحِمِ یهودا دِله زندگی کِردِنه. و خِشکسالی وِشون ره مجبور هَکِرده تا شه خاک ره ترک هَکِنِن و غریب جا یعنی موآبِ دِله کوچ هَکِنِن. 3امّا اَت کَمِ بعد نعومیِ مردی اِلیمِلِک بَمِرده و وِنه زِنا و وِنه دِ تا ریکا موآبِ دِله مونده‌گار بَینه. 4نَعومیِ دِ تا ریکا، دِ تا موآبی کیجای جا عروسی هَکِردِنه و حِدود ده سال باهم زندگی هَکِردِنه. اَتّایِ نوم عُرپَه و اَتّایِ دیگه نوم روت بی‌یِه. امّا ده سالِ بعد، 5نَعومی شه دِ تا ریکا رِم از دست هِدا، و نَعومی بَمونِسّه با دِ تا پِسِر زن.

\end{arab}


\section{acm}

Matthew 5:1-12 http://live.bible.is/bible/ACMAS3/MAT/5

\begin{arab}[utf]

فَلَمَّا رَأَى الْجُمْهُورَ صَعِدَ إِلَى الْجَبَلِ وَجَلَسَ. وَاقْتَربَ مِنْهُ تَلَامِيذُهُ،فَأَخَذَ يُعَلِّمُهُمْ وَقَالَ:"هَنِيئًا لِلْمَسَاكِينِ فِي الرُّوحِ، لأَنَّ لَهُمْ مَمْلَكَةَ اللهِ.هَنِيئًا لِلْحَزَانَى، لأَنَّهُمْ يَتَعَزَّوْنَ.هَنِيئًا لِلْوُدَعَاءِ، لأَنَّهُمْ يَرِثونَ الأَرْضَ.هَنِيئًا لِمَنْ يَجُوعُونَ وَيَعْطَشُونَ إِلَى الصَّلَاحِ، لأَنَّهُمْ يُشْبَعُونَ.هَنِيئًا لِلرُّحَمَاءِ، لأَنَّهُمْ يُرْحَمُونَ.هَنِيئًا لِمَنْ قُلُوبُهُمْ نَقِيَّةٌ، لأَنَّهُمْ يُشَاهِدُونَ اللهَ.هَنِيئًا لِمَنْ يَصْنَعُونَ السَّلامَ، لأَنَّهُمْ يُدْعَوْنَ أَبْنَاءَ اللهِ.هَنِيئًا لِمَنْ يَضْطَهِدُهُمُ النَّاسُ مِنْ أَجْلِ الصَّلَاحِ، لأَنَّ لَهُمْ نَصِيبًا فِي مَمْلَكَةِ اللهِ.هَنِيئًا لَكُمْ إِذَا شَتَمُوكُمْ وَاضْطَهَدُوكُمْ وَافْتَرَوْا عَلَيْكُمْ لأَنَّكُمْ أَتْبَاعِي،افْرَحُوا وَابْتَهِجُوا، لأَنَّ أَجْرَكُمْ فِي السَّمَاءِ عَظِيمٌ. فَإِنَّهُمُ اضْطَهَدُوا الأَنْبِيَاءَ الَّذِينَ قَبْلَكُم بِنَفْسِ الطَّرِيقَةِ. أنتم ملحُ ونُور
\end{arab}

\section{aeb}

Matthew 5:1-12 https://www.bible.com/bible/1304/MAT.5...

\begin{arab}[utf]
\section*{\textarab[utf]{الوَعْظَة فِي الجْبَلْ }}


وْكِي شَافْ يَسُوعْ النَّاسْ طْلَعْ لِلجْبَلْ وِوقْت اللّي قْعَدْ، قُربولو تْلامِذْتو، 2وِبْدَا يْعَلِّمْ فِيهُمْ، وْقَالْ:
3«صَحَّة لِيهُمْ المْسَاكِينْ فِي الرُّوحْ!
رَاهُمْ مَمْلِكْةْ السْمَاوَاتْ بَاشْ يَاخْذُوا.
4صَحَّة لِيهُمْ الحْزَانَى!
رَاهُمْ بَاشْ يِتْعَزُّوا.
5صَحَّة لِيهُمْ الطَّيْبِينْ!
رَاهُمْ الأَرْضْ بَاشْ يُورْثُوا.
6صَحَّة لِيهُمْ الجْوَاعَى وِالعْطَاشَى لِلحَقْ!
رَاهُمْ بَاشْ يِشْبْعُوا.
7صَحَّة لِيهُمْ الِّي يَرْحْمُوا!
رَاهُمْ بَاشْ يِتْرُحْمُوا.
8صَحَّة لِيهُمْ الِّي قْلُوبْهُمْ طَاهْرَة!
رَاهُمْ وِجِهْ الله بَاشْ يْشُوفُوا.
9صَحَّة لِيهُمْ الِّي يْصَلّْحُوا بِينْ النَّاسْ!
رَاهُمْ وْلاَدْ اللهْ بَاشْ يِتْسَمُّوا.
10صَحَّة لِيهُمْ الِّي يِتْعَذّْبُوا عْلَى خَاطِرْ الحَقْ!
مَمْلِكْةْ السْمَاوَاتْ بَاشْ يَاخْذُوا.
11صَحَّة لِيكُمْ وَقْتِلِّي النَّاسْ يْعَايْرُوكُمْ وِيْعَذّْبُوكُمْ وِيْقُولُوا عْلِيكُمْ الكْلاَمْ الخَايِبْ الكُلُّو عْلَى خَاطْرِي! 12أَفْرْحُوا وْهلّلُوا، رَاهُو أَجْرْكُمْ عْظِيمْ فِي السْمَاءْ. رَاهُو هَكَّا عَذّْبُوا الأَنْبِيَاءْ قْبَلْكُمْ.»
المِلْحْ وِالنُّورْ

\end{arab}

\section{apd}

Matthew 5:1-12 http://live.bible.is/bible/APDASV/MAT/5

\begin{arab}[utf]


5فَلَمَّا رَأَى الْجُمْهُورَ صَعِدَ إِلَى الْجَبَلِ وَجَلَسَ. وَاقْتَربَ مِنْهُ تَلَامِيذُهُ،فَأَخَذَ يُعَلِّمُهُمْ وَقَالَ:”هَنِيئًا لِلْمَسَاكِينِ فِي الرُّوحِ، لأَنَّ لَهُمْ مَمْلَكَةَ اللهِ. هَنِيئًا لِلْحَزَانَى، لأَنَّهُمْ يَتَعَزَّوْنَ.هَنِيئًا لِلْوُدَعَاءِ، لأَنَّهُمْ يَرِثونَ الأَرْضَ.هَنِيئًا لِمَنْ يَجُوعُونَ وَيَعْطَشُونَ إِلَى الصَّلَاحِ، لأَنَّهُمْ يُشْبَعُونَ.هَنِيئًا لِلرُّحَمَاءِ، لأَنَّهُمْ يُرْحَمُونَ.هَنِيئًا لِمَنْ قُلُوبُهُمْ نَقِيَّةٌ، لأَنَّهُمْ يُشَاهِدُونَ اللهَ.هَنِيئًا لِمَنْ يَصْنَعُونَ السَّلامَ، لأَنَّهُمْ يُدْعَوْنَ أَبْنَاءَ اللهِ.هَنِيئًا لِمَنْ يَضْطَهِدُهُمُ النَّاسُ مِنْ أَجْلِ الصَّلَاحِ، لأَنَّ لَهُمْ نَصِيبًا فِي مَمْلَكَةِ اللهِ.هَنِيئًا لَكُمْ إِذَا شَتَمُوكُمْ وَاضْطَهَدُوكُمْ وَافْتَرَوْا عَلَيْكُمْ لأَنَّكُمْ أَتْبَاعِي،افْرَحُوا وَابْتَهِجُوا، لأَنَّ أَجْرَكُمْ فِي السَّمَاءِ عَظِيمٌ. فَإِنَّهُمُ اضْطَهَدُوا الأَنْبِيَاءَ الَّذِينَ قَبْلَكُم بِنَفْسِ الطَّرِيقَةِ.
\end{arab}

\section{arq}

Matthew 5:1-12 https://ecritures-algerie.com

\begin{arab}[utf]
\section*{\textarab[utf]{الخَطبة مْتاع الجْبَل – يا سَعد}}


51 كي شاف يَسوع الغاشي، طْلَع للجْبَل. وكي قْعَد، أَدَّناو ليه التابعين مْتاعو، 2 وبَدا يْدَرَّس فيهُم وقال: 3 "يا سَعد الڤْلالين فالروح خاطَر
مَلَكوت السْما ليهُم. 4 يا سَعد الحْزانى خاطَر رايحين يَتعَزّاو. 5 يا سَعد الحْنان خاطَر رايحين يَوَّرتو ا لأَرض. 6 يا سَعد الجيعانين
والعَطشانين للصْلاح، خاطَر رايحين يَشَّبعو، 7 يا سَعد الرُحَما خاطَر رايحين يْنالو الرَحمة، 8 يا سَعد اللي قَلبهُم صافي خاطَر رايحين يْشوفو الله،
9 يا سَعد اللي يْسَبّبو السْلام خاطَر رايحين يَتسَمّاو "وْلاد الله"، 10 يا سَعد اللي راهُم مَحڤورين على جال الصْلاح، خاطَر مَلكوت السْما ليهُم، 11 يا
سَعدكُم كُل ما يْعايروكُم ويَحَّڤروكُم، ويْقولو عليكُم بالكْدَب كُل دوني على جالي، 12 أَفَّرحو و أَسَّعدو خاطَر آجَركُم كْبير فالسْما، خاطَر هَكدا حَڤرو
ا لأَنبيا اللي كانو قْبَلكُم.
\end{arab}





\section{ary}

Matthew 5:1-12 https://www.bible.com/bible/558/MAT.5.MSTD

\begin{arab}[utf]
\section*{\textarab[utf]{الْخُطْبَة دْيَالْ يَسُوعْ فُوقْ الجّْبَلْ}}


وْمْلِّي شَافْ يَسُوعْ الجّْمَاعَاتْ دْ النَّاسْ طْلَعْ لْلجّْبَلْ، وْمْلِّي ݣْلَسْ جَاوْ لْعَنْدُه التّْلَامْدْ دْيَالُه، 2وْبْدَا كَيْعَلّْمْهُمْ وْݣَالْ:
3«سْعْدَاتْ الْفُقَرَا فْالرُّوحْ،
حِيتْ لِيهُمْ مَمْلَكَةْ السَّمَاوَاتْ.
4سْعْدَاتْ اللِّي كَيْبْكِيوْ، حِيتْ اللَّهْ غَادِي يْوَاسِيهُمْ.
5سْعْدَاتْ الْمْتْوَاضْعِينْ، حِيتْ غَادِي يْوَرْتُو الْأَرْضْ.
6سْعْدَاتْ هَادُوكْ اللِّي جِيعَانِينْ وْعْطْشَانِينْ لْطَاعْةْ اللَّهْ، حِيتْ غَادِي يْشْبْعُو.
7سْعْدَاتْ هَادُوكْ اللِّي كَيْرَحْمُو، حِيتْ اللَّهْ غَادِي يْرْحَمْهُمْ.
8سْعْدَاتْ هَادُوكْ اللِّي قَلْبْهُمْ نْقِي، حِيتْ غَادِي يْشُوفُو اللَّهْ.
9سْعْدَاتْ هَادُوكْ اللِّي كَيْعَاوْنُو النَّاسْ يْعِيشُو فْالْهْنَا، حِيتْ غَادِي يْتْسَمَّاوْ وْلَادْ اللَّهْ.
10سْعْدَاتْ هَادُوكْ اللِّي كَيْتّْعَدَّاوْ عْلِيهُمْ عْلَاحْقَّاشْ تَابْعِينْ طْرِيقْ الْحَقّْ، حِيتْ لِيهُمْ مَمْلَكَةْ السَّمَاوَاتْ.
11وْسْعْدَاتْكُمْ إِلَا عَايْرُوكُمْ النَّاسْ وْتّْعَدَّاوْ عْلِيكُمْ وْݣَالُو فِيكُمْ ݣَاعْ الْهَضْرَة الْخَايْبَة بْالْكْدُوبْ، عْلَى وْدِّي. 12فْرْحُو وْطِيرُو بْالْفَرْحَة عْلَاحْقَّاشْ أَجْرْكُمْ عْظِيمْ فْالسَّمَاوَاتْ، حِيتْ رَاهْ بْحَالْ هَكَّا تْعَدَّاوْ عْلَى الْأَنْبِيَا اللِّي قْبَلْ مْنّْكُمْ».
\end{arab}

\section{ayp}

Matthew 5:1-12 http://live.bible.is/bible/AYPABT/MAT/5

\begin{arab}[utf]

وْلَمَنْ أَرى يَسوُعْ لَمّات النّاِسْ، طَلَعْ للجَّبَلْ. وْلَمَنْ قعدْ، قَرَّبوا لَ عَنْدو تَلاميذو،وْفَتَح ثمّو يعَلّمنْ، قالْ:”هَنيّة المحْتازينْ لَالله بالرّوُحْ: لهنْ واِ مَلَكوت السَّماواتْ.هَنيّة الحَزانىِ: هنّاِ تَ يتْعَزَّونْ.هَنيّة المتْواضْعينْ: هنّاِ تَ يورَثون الاَرْضْ.هَنيّة الجَّواعىِ والعَطاشيِ للبرْ: هنّاِ تَ يشْبَعونْ.هَنيّة الرَّحومينْ: عَلَينْ تَ تْكونْ الرَّحْمِة.هَنيّة لَ قَلْبنْ طاهرْ: هنّاِ تَ يرَون اَلله.هَنيّة لَ يسَون سَلام: اولاد اَلله تَ يتْسَمَّونْ.هَنيّة لَ ينْطهْدونْ لَخاطرْ البرْ: لهنْ واِ مَلَكوت السَّماواتْ.”هَنيّتكنْ لَمَنْ يهينوكن وْيضْطهْدوكنْ ويقولونْ عَلَيكنْ كلْ كلْمِة عاطْلِة بالگذْبْ لَخاطري.هاك الوَقْتْ افْرَحوا وْهَلْهلوا، منْ لَ أَجرْكن گْبير واِ ف السَّماواتْ، منْ لَ كذاِ اضْطَهَدوا الاَنْبِيا قَبلْكنْ.
\end{arab}



\section{bcc}

Matthew 5:1-12 https://www.bible.com/bible/1498/MAT.5.HPKB

\begin{arab}[utf]
\section*{\textarab[utf]{بَهتاوری }}


مردمانی مزنێن مُچّیئے گِندگا رند، ایسّا کۆهێئے سرا سر کپت. وهدے بُرزگا نِشت، مرید هم آییئے کِرّا آتکنت. 2گڑا اے ڈئولا آیانی تالیم دئیگا لگّت:
3«بَهتاور اَنت هما که آیانی روه وار و بَزّگ اِنت، چیا که آسمانی بادشاهی همایانیگ اِنت.
4«بَهتاور اَنت هما که گَمیگ و پُرسیگ اَنت، چیا که آیان تسلّا و دِلبڈّیَ رَسیت.
5«بَهتاور اَنت هما که نرم‌دل و مهربان اَنت، چیا که زمینئے میراس بَرۆک هما بنت.
6«بَهتاور اَنت هما که هَکّ و اِنساپئے شدیگ و تُنّیگ اَنت، چیا که آ سێرَ بنت.
7«بَهتاور اَنت هما که په دگران رهمَ کننت، چیا که آیانی سرا هم رهم کنگَ بیت.
8«بَهتاور اَنت هما که دِلِش پاک و ساپ اِنت، چیا که آ هُدایا گندنت.
9«بَهتاور اَنت هما که سُهل و اێمنیَ کارنت، چیا که آ هُدائے چُکّ نامێنگَ بنت.
10«بَهتاور اَنت هما که په هُدائے راهئے گِرَگا بُهتام جنَگ و آزار دئیگَ بنت، چیا که آسمانی بادشاهی همایانیگ اِنت.
11«بَهتاور اێت شما که په منیگی مردم شمارا دُژمانَ دئینت و آزارَ رسێننت، شمئے پُشتا درۆگَ بندنت و بدَ گوَشنت. 12گَل و وَشدل ببێت و شادهی بکنێت، چیا که آسمانا شمئے مُزّ سکّ باز اِنت. پێسریگێن نبیانی سرا هم اے ڈئولێن آزارِش رسێنتگ.

\end{arab}

\section{fas}

Matthew 5:1-12 http://live.bible.is/bible/PESTPV/MAT/5

\begin{arab}[utf]
\section*{\textarab[utf]{موعظهٔ سر كوه }}


وقتی عیسی جمعیّت زیادی را دید، به بالای كوهی رفت و در آنجا نشست و شاگردانش به نزد او آمدندو او دهان خود را گشوده به آنان چنین تعلیم داد:

\section*{\textarab[utf]{خوشبختی واقعی}}

«خوشا به حال کسانی‌که از فقر روحی خود آگاهند
زیرا، پادشاهی آسمان از آن ایشان است.
«خوشا به حال ماتم‌زدگان،
زیرا ایشان تسلّی خواهند یافت.
«خوشا به حال فروتنان،
زیرا ایشان مالک جهان خواهند شد.
«خوشا به حال کسانی‌که گرسنه و تشنهٔ نیكی خدایی هستند،
زیرا ایشان سیر خواهند شد.
«خوشا به حال رحم‌كنندگان،
زیرا ایشان رحمت را خواهند دید.
«خوشا به حال پاکدلان،
زیرا ایشان خدا را خواهند دید.
«خوشا به حال صلح‌كنندگان،
زیرا ایشان فرزندان خدا خوانده خواهند شد.
«خوشا به حال کسانی‌که در راه نیكی آزار می‌بینند،
زیرا پادشاهی آسمان از آن ایشان است.

«خوشحال باشید اگر به‌خاطر من به شما اهانت می‌کنند و آزار می‌رسانند و به ناحق هرگونه افترایی به شما می‌زنند.خوشحال باشید و بسیار شادی كنید، زیرا پاداش شما در آسمان عظیم است، چون همین‌طور به انبیای قبل از شما نیز آزار می‌رسانیدند.
\end{arab}

\section{fuv}

Matthew 5:1-12 https://www.bible.com/en-GB/bible/2377/MAT.5.FUVASNT

\begin{arab}[utf]
\section*{\textarab[utf]{يٜىٰسُ وَعَجِنِ دٛوْ پٛلْدٜ}}

ندٜ يٜىٰسُ يِعِ يِمْٻٜ طُطْٻٜ نغَرِ، سٜيْ اٛ پٜنْتِ پٛلْدٜ اٛ جٛوطِ، تٛکُّٻٜمٛ نغَرِ تٛ مَاکٛ؞ 2سٜيْ اٛ فُطِّ اٜکِّتِنْکِٻٜ، اٛ وِعِ،

\section*{\textarab[utf]{بَرْکِطِنْکِ}}

«‏ٻٜ بَرْکِطِنَاٻٜ غَطُٻٜ فُو کَمُّندٜ دٛوْ اَللَّه،
نغَمْ کَمْٻٜ نجٜيِ لَامُ اَللَّه؞
4‏«‏ٻٜ بَرْکِطِنَاٻٜ يِمْٻٜ وٛيٛوٻٜ،
نغَمْ طُمْ وَلْتِنَيْ ٻٜرْطٜ مَٻّٜ؞
5‏«‏ٻٜ بَرْکِطِنَاٻٜ لٜيْنٛوٻٜ کٛعٜ،
نغَمْ کَمْٻٜ ندٛنَتَ دُونِيَارُ؞
6‏«‏ٻٜ بَرْکِطِنَاٻٜ نَنٛوٻٜ وٜىٰلٛ
اٜ طٛنْکَ وَطُکِ اَادِلَاکُ،
نغَمْ ٻٜ کَرْنٜتٜىٰٻٜ؞
7‏«‏ٻٜ بَرْکِطِنَاٻٜ جُرْمِنَنٛوٻٜ يِمْٻٜ،
نغَمْ کَمْٻٜ مَا طُمْ يُرْمِنَنَيْٻٜ؞
8‏«‏ٻٜ بَرْکِطِنَاٻٜ ٻٜ ٻٜرْطٜ لَٻْطٜ،
نغَمْ ٻٜ نغِعَيْ اَللَّه؞
9‏«‏ٻٜ بَرْکِطِنَاٻٜ شِرْيٛوتِرٛوٻٜ يِمْٻٜ،
نغَمْ طُمْ نٛدِّرَيْٻٜ ٻِٻّٜ اَللَّه؞
10‏«‏ٻٜ بَرْکِطِنَاٻٜ تٛرّٜتٜىٰٻٜ نغَمْ وَطُکِ
اَادِلَاکُ،
نغَمْ کَمْٻٜ نجٜيِ لَامُ اَللَّه؞
11‏«‏اٛنْ بَرْکِطِنَاٻٜ تٛ يِمْٻٜ کُطِيعٛنْ، تٛرِّيعٛنْ، ڤٜوَنِيعٛنْ مبِعِي حَالَاجِ کَلُّطِ فٜىٰرٜ‑فٜىٰرٜ دٛوْ مٛوطٛنْ نغَمْ اَمْ؞ 12نَنٜىٰ بٜلْطُمْ، مبٜلْمبٜلْتٜىٰ، نغَمْ بَرْکَ مٛوطٛنْ طُمْ طُطُّمْ اٜ دٛوْ، نغَمْ نٛنْ ٻٜ نغَطَنّٛو اَنَّبٛعٜنْ اَرَنْدٜعٜنْ؞

\end{arab}

\section{glk}

Matthew 5:1-12 http://live.bible.is/bible/GLKGMV/MAT/5

\begin{arab}[utf]

وختی کی عیسی بیده مردومأنٚ زیادی بَموییدی، بوشو ایتأ کوه بولندی سر بینیشته و اونٚ شاگردأنم بَمویید اونٚ ورجهو عیسی شروع بوکوده به تعلیم دَئن و بوگفته:«خوشا بحالٚ اوشأنی کی خوشأنٚ نیازَ به خودا احساس کونیدی، چونکی آسمانٚ پادشاهی اوشأنٚ شینه.خوشا بحالٚ کسأنی کی ماتم بزه‌کیدی چونکی اوشأنم آرامش پیدا کونیدی.خوشا بحالٚ فروتنأن، چونکی جهان اوشأنٚ شین به.خوشا بحالٚ اوشأنی کی عیدالتٚ وسی گوشنه و تشنه ایسیدی چونکی اوشأن سِئرَ بیدی.خوشا بحالٚ اوشأنی کی رحم و مروت دأریدی، چونکی اوشأن خودا جَا رحم و مروت دینیدی.خوشا بحالٚ کسأنی کی اوشأنٚ دیل پاکه چونکی اوشأن خودایَ دینیدی.خوشا بحالٚ کسأنی کی صلح طلب ایسیدی، چونکی اوشأن خودا زأکأن دوخوأده بیدی.خوشا بحالٚ اوشأنی کی خوشأنٚ درستکاری وسی اذیت و آزار دینیدی، چونکی آسمانٚ پادشاهی اوشأنٚ شینه.«خوشا بحالٚ شومأن وختی کی مردومأن می وسی شمرَ فحش و فلاکت دیهیدی و اذیت و آزار کونیدی و شمرَ همه جوره بوهتأن زنیدی.خوشحال بیبید و شادی بوکونید، چونکی بَزین آسمانٚ جور پیله پاداش شیمی شین به، چره کی هَطویم پیغمبرأنی کی پیشتر بَموییدی اذیت و آزار بیده‌ییدی.

\end{arab}


\section{kmr}

Matthew 5:1-12 http://live.bible.is/bible/KMRKLA/MAT/5

\begin{arab}[utf]
%\addfontfeature{CharacterVariant={48:2}}
%\testfeature{Language=Kurdish}
\fontspec{Lateef}[Script=Arabic,Language=Kurdish,Scale=1.5]

\section*{\textarab[utf]{گۆتارا ل سەرێ چیای }}

دەمێ عیسای جەماوەرێ خڕڤەبووی دیتی، سەركەفتە چیای و روینشت. شاگردێن وی خۆ نێزیكی وی كر

و وی دەست ب فێركرنا وان كر و گۆت:
«خوزیكێن وان یێن ب گیانی د هەژار،
چنكو پاشایەتیا ئەسمانان یا وانە.
خوزیكێن خەمباران،
چنكو دێ هێنە دلشادكرن.
خوزیكێن دەروونبچویكان،
چنكو ئەو دێ بنە میراتگرێن ئەردی.
خوزیكێن برسی و تێهنیێن راستداریێ،
چنكو ئەو دێ هێنە تێركرن.
خوزیكێن دلۆڤانان،
چنكو دێ دلۆڤانی پێ هێتە برن.
خوزیكێن دلپاقژان،
چنكو ئەو دێ خودێ بینن.
خوزیكێن ئاشتیخوازان،
چنكو دێ بێژنە وان كوڕێن خودێ.
خوزیكێن وان یێن ژ بەر راستداریێ تەپەسەر دبن،
چنكو پاشایەتیا ئەسمانان یا وانە.

خوزیكێن هەوە، دەمێ ژ بەر من هەوە فهێت دكەن و تەپەسەر دكەن و ژ درەو ڤە هەمی جۆرێن پەیڤێن خراب دژی هەوە دبێژن.دلخۆش و دلشاد ببن! ژ بەر كو خەلاتێ هەوە ل ئەسمانان یێ مەزنە، چنكو وان بەری هەوە پێغەمبەر ژی هۆسا تەپەسەر دكرن.

\end{arab}


\section{pbt}

Matthew 5:1-12 https://pashtobibles.org/matthew/matthew-5

\begin{arab}[utf]
\section*{\textarab[utf]{په غر دپاسه د حضرت عيسىٰ بيان }}



۱کله چې عيسىٰ ګڼ خلق وليدل نو غرۀ ته وختلو او هلته کښېناستو. او کله چې د هغۀ مريدان ورته راټول شول، ۲نو هغۀ ورته داسې تعليم شروع کړو، ۳”بختور دى هغوئ چې د روح غريبان دى ځکه چې د آسمان بادشاهى د هغوئ ده. ۴بختور دى هغوئ چې غمژن دى، هغوئ به تسلى ومومى. ۵بختور دى هغوئ چې حليمان دى ځکه چې دوئ به د زمکې وارثان شى. ۶بختور دى هغوئ چې د صداقت اوږى تږى دى، دوئ به ماړۀ کړے شى. ۷بختور دى هغوئ چې رحم کوى، په هغوئ به رحم وکړے شى. ۸بختور دى هغوئ چې زړونه يې پاک دى، هغوئ به د خُدائ پاک ديدن وکړى. ۹بختور دى د امن راوستونکى، خُدائ پاک به هغوئ ته خپل زامن ووائى. ۱۰بختور دى هغوئ چې د خُدائ پاک د رضا پوره کولو دپاره زورَولے کيږى، د آسمان بادشاهى د هغوئ ده. ۱۱تاسو بختور يئ کله چې خلق درته سپکې سپورې وائى، او زوروى مو او زما له وجې خلق ستاسو په حقله هر قِسم دروغژنې بدې خبرې کوى. ۱۲دا هر څۀ تاسو په خوشحالۍ او روڼ تندى وزغمئ ځکه چې په آسمان کښې ستاسو دپاره لوئ اجر دے، ځکه چې ستاسو نه وړاندې هم دغه رنګ دوئ نبيان وزورول.

\end{arab}


\section{pnb}

Matthew 5:1-12 https://www.bible.com/en-GB/bible/2912/MAT.5.PAN2020

\begin{arab}[utf]
\section*{\textarab[utf]{پہاڑی خُطبہ }}

تے بِھیڑ نُوں ویکھ کے اوہ پہاڑ اُتّے چڑھ گیا تے جد بَہہ گیا اوہدے شاگِرد اوہدے کول آئے۔ 2تے اوہ اپنا مُنھ کھول کے اوہناں نُوں ایہہ آکھ کے سکھان لگّا۔

\section*{\textarab[utf]{دَھنّی لوک}}

دھنّ نیں اوہ جیہڑے دِل دے غرِیب نیں
کیوں جو اسمان دی بادشاہی اوہناں دی اے۔
4دھنّ نیں اوہ جیہڑے غمگِین نیں
کیوں جو اوہناں نُوں تسلّا دِتّی جائے گی۔
5دھنّ نیں اوہ جیہڑے حلِیم نیں
کیوں جو اوہ زمین دے وارث ہون گے۔
6دھنّ نیں اوہ جیہڑے راستبازی دے بُھکھے تے تریہائے نیں
کیوں جو اوہ رجائے جان گئے۔
7دھنّ نیں اوہ جیہڑے رحمدِل نیں
کیوں جو اوہناں اُتّے رَحم کِیتا جائے گا۔
8دھنّ نیں اوہ جیہڑے دِل دے صاف نیں
کیوں جو اوہ خُدا نُوں ویکھن گے۔
9دھنّ نیں اوہ جیہڑے صُلح کران والے نیں
کیوں جو اوہ خُدا دے پُتر اکھوان گے۔
10دھنّ نیں اوہ جیہڑے راستبازی دی خاطر ستائے گئے نیں
کیوں جو اسمان دی بادشاہی اوہناں دی اے۔
11دھنّ ہو تُسی جد میری خاطر تُہانوں ملامت کرن گے تے ستان گے تے جُھوٹھ مُوٹھ ہر طرح دِیاں بُریائِیاں تُہاڈے اُتّے بکن گے۔ 12خُوش ہو تے چھالاں مارو کیوں جو اسمان وِچ تُہاڈا بدلہ بڑا اے۔ ایس کر کے پئی اوہناں پیغمبراں نُوں جیہڑے تُہاڈے توں اگّے سن ایسے طرح ستایا سی۔

\end{arab}

\section{rif}

Matthew 5:1-12 https://live.bible.is/bible/RIFNVS/MAT/5...

\begin{arab}[utf]

وَامِي يژْرَا قَاع ڒْغَاشِي نِّي، يُوْڒِي غَار وذْرَار يقِّيم ؤُسِين-د غَارس يمحْضَارن نّس.ئِبْذَا يسڒْمَاذ-يثن ينَّا-ٱسن:"سَّعْذ نْسن إِ يِنِّي يدْجَان ذ ڒْمُسَاكِين ذݣ اَرُّوْح نْسن، أَقَا ڒْمُلْك وجنَّا إِ نِثْنِي.سَّعْذ نْسن إِ يِنِّي يتْرُوْن، أَقَا اَذ تْوَاشجّْعن.سَّعْذ نْسن إِ يِنِّي يدْجَان ذ دْرَاوش، أَقَا اَذ وَرْثن دُّنشْت.سَّعْذ نْسن إِ يِنِّي يدْجُوْزن يفُّوْذن غَار مِين يدْجَان نِيشَان، ڒَاحقَّاش نِثْنِي اَذ جَّاوْنن.سَّعْذ نْسن إِ يِنِّي يرحّْمن، أَقَا اَثن يرْحم أَربِّـي ؤُڒَا ذ نِثْنِي.سَّعْذ نْسن إِ يِنِّي غَار يدْجَا وُوْڒ ذ اَشمْڒَاڒ، أَقَا اَذ ژرن أَربِّـي.سَّعْذ نْسن إِ يِنِّي يصدْجْحن يوْذَان، أَقَا نِثْنِي اَثن سمَّان اَمشْنَاو ثَارْوَا ن أَربِّـي.سَّعْذ نْسن إِ يِنِّي يتْوَاعدَّابن خ سَّبَاب ن مِين يدْجَان نِيشَان، أَقَا ڒْمُلْك ن وجنَّا إِ نِثْنِي.سَّعْذ نْوم مڒَا تْزَوَارن-كنِّيو، عدّْبن-كنِّيو، نِغ نَّان ذَايْوم ڒْعِيب س يخَارِّيقن خ سَّبَاب ينُو.ضحْشث، فَرْحث. أَقَا غَارْوم يجّن ن لْأَجَار ذ اَمقّْرَان ذݣ وجنَّا ڒَاحقَّاش اَمنِّي إِي تْوَاعدّْبن لْأَنْبِيَا تُوْغَا يدْجَان قْبڒ نْوم.

\end{arab}

\section{snd}

Matthew 5:1-12 https://www.bible.com/bible/722/MAT.5.SCLNT

\begin{arab}[utf]

\addfontfeature{CharacterVariant={44:0}}
\addfontfeature{CharacterVariant={84:0}}

\section*{\textarab[utf]{ٽڪر تي وعظ }}


جڏھن عيسيٰ ميڙ ڏٺا تہ ھو ٽڪر تي چڙھي ويو ۽ شاگرد سندس چوڌاري مڙي آيا. 2ھو انھن کي ويھي تعليم ڏيڻ لڳو ۽ چيائين تہ ”سڀاڳا آھن اھي جيڪي دل جا نماڻا آھن،
ڇالاءِ⁠جو اھي آسمان واري بادشاھت جون برڪتون ماڻين ٿا.
4 سڀاڳا آھن اھي جيڪي ڏکويل آھن،
ڇالاءِ⁠جو خدا انھن کي تسلي ڏيندو.
5 سڀاڳا آھن اھي جيڪي تابعدار آھن،
ڇالاءِ⁠جو خدا انھن کي زمين جو وارث ڪندو.
6 سڀاڳا آھن اھي جيڪي سچائيءَ جا بکايل ۽ اڃايل آھن،
ڇالاءِ⁠جو خدا انھن کي چڱيءَ طرح ڍءُ ڪرائيندو.
7سڀاڳا آھن اھي جيڪي ٻين تي رحم ڪن ٿا،
ڇالاءِ⁠جو خدا انھن تي رحم ڪندو.
8 سڀاڳا آھن اھي جيڪي دل جا صاف آھن،
ڇالاءِ⁠جو اھي خدا کي ڏسندا.
9سڀاڳا آھن اھي جيڪي ماڻھن ۾ صلح ڪرائين ٿا،
ڇالاءِ⁠جو خدا انھن کي پنھنجو ٻار سڏيندو.
10 سڀاڳا آھن اھي جيڪي سچائيءَ جي ڪري ستايا وڃن ٿا،
ڇالاءِ⁠جو اھي آسمان واري بادشاھت جون برڪتون ماڻين ٿا.
11 سڀاڳا آھيو اوھين، جو ماڻھو منھنجي ڪري اوھان جي بي⁠عزتي ڪن، اوھان کي ستائين ۽ اوھان تي ھر قسم جون تھمتون ھڻن، 12 ڇوتہ اھڙيءَ طرح ئي نبين کي بہ ستايو ويو ھو. خوش ٿيو ۽ ڏاڍا سرھا ٿيو، ڇاڪاڻ⁠تہ اوھان کي بھشت ۾ وڏو اجر ملندو.“


\end{arab}

\section{tuk}

Matthew 5:1-12 https://www.scriptureearth.org/data/tuk/sab/tuk-41-MAT-005.html

\begin{arab}[utf]

\section*{\textarab[utf]{حاقیقی خوشواقتلیق  }}

‏1 عیسی اویشِن جِماغاتی گُروپ، داغا چیقیپ اُتوردی.‏ شَگیرتلِری اُنونگ یانینا گِلدیلِر، ‏2 عیسی هِم اُلارا اُورِتمَگه باشلادی:‏

‏3 ‏«‏روحی قاریپلار خوشواقتدیر، 
چونکی آسمانینگ پادیشاهلیغی اُلارینگقیدیر.‏
‏4 یاس توتیانلار خوشواقتدیر،
چونکی اُلارا تِسِلّی بِریلِر.‏
‏5 فُروتَن آداملار خوشواقتدیر،
چونکی اُلار یِر یوزونی میراث آلارلار.‏
‏6 عادالاتلیغا آجیغیپ-سوسانلار خوشواقتدیر،
چونکی اُلار دُیارلار.‏
‏7 رِحم اِدیَنلِر خوشواقتدیر،
چونکی اُلارا-دا رِحم اِدیلِر.‏
‏8 قالبی پَکلِر خوشواقتدیر،
چونکی اُلار خودایی گُرِرلِر.‏
‏9 پاراحاتلیق دُرِدیَنلِر خوشواقتدیر،
چونکی اُلارا خودایینگ اُغول‍لاری دییلِر.‏
‏10 تاقوالیق اوچین آزارلانانلار خوشواقتدیر،
چونکی آسمانینگ پادیشاهلیغی اُلارینگقیدیر.‏


‏11 ‏«‏مِن سِبَپلی سِوولِن، آزارلانان و ناحاق یِره هِر هیلی غیباتینگیز اِدیلِن ماحالی سیز خوشواقتسینگیز!‏ ‏12 شاتلانینگ، بِگِنینگ، سِبَبی آسماندا آلجاق سیلاغینگیز اولودیر.‏ سیزدِن اُنگ پیغامبارلار-دا شِیدیپ آزارلاندیلار.‏ 

\end{arab}

\section{tzm}

Matthew 5:1-12 http://live.bible.is/bible/TZMWYI/MAT/5

\begin{arab}[utf]


‏1 ألّيݣ يانّاي سيدنا عيسى شيݣان ن ميدّن يالي غر عاري، ئطّارح، دّوند غورس ئمحضرن نّس. ئكّر أر تن-ئسّقرا أراسن ئتّيني:"أياعري نّون أييمژلاض ئدّخ تݣا تݣلديت ن ربّي تينّون. أياعري نّسن يي وينّا ئحزنن، ئدّخ دّان أد تّوعزان. أياعري-نّسن ئ-وينّا ئتواضعن، ئدّخ دّان أد كّوسن أشال. أياعري نّسن يي وينّا ݣ ئلّا لاژ د فاد ن-ؤبريد ن ربّي، ئدّخ دّان أتافن. أياعري نّسن يي وينّا ݣ تلّا رّحمت، ئدّخ دّان أد تّورحمن. أياعري نّسن يي وينّا مي ئصفا ؤول، ئدّخ دّان أد أنّاين ربّي. أياعري نّسن ئ-وينّا ئزرّعن ئفسان ن لهنا، ئدّخ دّان أد تّوسمان أرّاو ن ربّي. أياعري نّسن ئ-وينّا ئتّوعدّابن ئيتدخ تفارن أبريد ن لمسيح، ئدّخ تينسن أيتݣا تݣلديت ن-ئݣنوان. أياعري-نّون مش كون رݣمن، مرميدّن كون أها ئنين غيفون أوال ؤر ئحلين ئݣان ئحلّالن خف نّيرت-ئنو نكّين. فرحات ترشقاون، ئدّخ ئدّا أد ئمغور لاجر نّون غر ربّي، هاثين ئمشيد أيد مرمدن ميدّن لانبييا نّا زرينين


\end{arab}


\section{bqi}

John 1:1-9 http://live.bible.is/bible/BQIBMV/JHN/1

\begin{arab}[utf]

\section*{\textarab[utf]{کلُم خݚا قالوٚ آݚُم به خوس اِگِره }}


1زِ هَمو اولِس کلُمِ بیݚ، و کلُم وابا خݚا بیݚ و کلُم خوݚِ خݚا بیݚ.
2هو زِ هَمو اول وابا خݚا بیݚ.
3همه چی زِ پِݚو هو وافِریݚه وابیݚ و زِ همه چیاییٚ که وافریݚه وابیݚِن، چی نَبیݚ که هو نَوافریݚه بوس.

4مِن وُجیݚِ هو زِندیی بیݚ، و او زِندییُ سی آݚُمیَل روشنایی بیݚ.
5او روشنایی‌ُ مِن تاریکی بِرچ اِزَنه و تاریکی سر زِس نَدِرَوُرده.

6پیاییٚ اوَیݚ که خݚا فِشناݚِس بی؛ و اسمِس یحیی بیݚ.
7هو اوَیݚ بی که شهاݚت بِݚه، شهاݚت بِݚه که ای روشنایی‌ُ اِیاهه، تا همه زِ پِݚو هو ایموٚۨ بیارِن.
8یحیی خوس او روشناییُ‌ نَبیݚ. هو اوَیݚ تا راجو او روشناییُ‌‌ شهاݚت بِݚه.
9هَمو روشنایی که حقیٚقت بیݚ و دلِ آݚُمِ روشنا اِکُنه، اوَیݚ به ای دنیاهو.

\end{arab}



\end{document}